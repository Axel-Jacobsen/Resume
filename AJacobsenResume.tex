% Original template author:
% Debarghya Das (http://debarghyadas.com)
%
% Original repository:
% https://github.com/deedydas/Deedy-Resume
%
% Modifications and my personal history:
% Me

\documentclass[]{deedy-resume-openfont}
\setlength{\parskip}{.4em}
\thispagestyle{empty}

\begin{document}

\namesection{}{Axel Jacobsen}{\href{https://axel-jacobsen.github.io}{axel-jacobsen.github.io} | \href{mailto:axelnjacobsen@gmail.com}{axelnjacobsen@gmail.com}}

\begin{minipage}[t]{\textwidth}
  \runsubsection{Languages} Advanced in \custombold{Python}, experienced in \custombold{C}, \custombold{Java}, \custombold{Javascript}, learning \custombold{Julia}, \custombold{Lisp}, \custombold{Rust}. Experienced in \custombold{git}\\
  \ifdef{\software} {
    \runsubsection{Deep Learning} Used LSTMs, CNNs, and various RL techniques (Asynchronous-Advantage Actor Critic, DQNs) in Pytorch
  } {
    \ifdef{\robotics} {
      \runsubsection{Electromechanical} Experience with robotics, PCB design, mechanical prototyping and Onshape/Solidworks
    } { 
      \runsubsection{Deep Learning} Used LSTMs, CNNs, and various RL techniques (Asynchronous-Advantage Actor Critic, DQNs) in Pytorch \\
      \runsubsection{Electromechanical} Experience with robotics, PCB design, mechanical prototyping and Onshape/Solidworks
    }
  }
\end{minipage}

\vspace{12pt}

\section{Relevant Experience}

\runsubsection{Chan-Zuckerberg Biohub}
\descript{| R\&D Engineering Intern | \small{Jun 2020 - Dec 2020, Jul 2021 - Present}}
\begin{tightemize}
    \item Rewrote codebase of the \href{https://opentrons.com/ot-2/}{Opentrons OT-2}, an open-source pipetting robot
    \item Doubled the speed of certain frequent operations (e.g. picking up / dropping pipette tips)
    \item Reduced codebase size by \char`\~ 75\% while maintaining previous functionality, simplifying code for future maintenance / development
    \item Designed and implemented architecture of robot / software so any software-controlled instrument can be used during protocols (e.g. cameras, GPUs, thermocyclers)
    \item Created tests, accurate to 1 µm over \char`\~ 1 meter, to measure the drift of the OT-2 using cross-correlation techniques
\end{tightemize}

\vspace{8pt}

\runsubsection{Wildlife Computers}
\descript{| Engineering Intern | \small{May 2019 – Aug 2019} }
\begin{tightemize}
    \item Designed an isolator PCB to isolate digital lines from sensitive measurement devices, allowing low-noise and accurate voltage measurements
    \item Wrote C++ to test PCBs that arrive from fabrication - autonomously verifies PCB component placement to increase production throughput
\end{tightemize}

\vspace{8pt}


\runsubsection{Control Mobile}
\descript{| Data Science Co-op | \small{Jan 2018 – Apr 2018} }
\begin{tightemize}
    \item Wrote Python scripts to analyze and rank order over 300 individual SQL queries by their runtime to optimize the SQL database; reduced the runtime to fetch and display customer data by 65\%
    \item Worked with the backend team to fix existing bugs, write new code, and refactor current code
\end{tightemize}

\vspace{12pt}


\section{Projects}

\runsubsection{Deep Learning}
\begin{tightemize}
    \item Asynchronous Advantage Actor-Critic Model written in Pytorch, optimized for multicore CPUs via multiprocessing
    \item LSTM-based Deep Q-Network, trained on Denmark Technical University's High-Performance Computing Cluster
    \item Feed-forward neural network written from scratch, implementing the math behind deep learning
    \item Currently writing a basic autograd library in Julia, in order to understand fundamentals of Pytorch
\end{tightemize}

\vspace{8pt}

\runsubsection{Engineering Physics Autonomous Robot Competition}
\begin{tightemize}
    \item Designed and created an autonomous robot from scratch in 8 weeks, capable of navigating complex and dynamic course
    \item Implemented signal processing software to detect specific IR frequencies with sub-millisecond detection time
    \item Designed and created circuits to control the mechanical subsystems (robotic arm / claw)
    \item Wrote (in C) driver software for the robotic arm / claw, as well as software for high-level control loops of robot
\end{tightemize}

\vspace{12pt}


\section{Education}

\runsubsection{University of British Columbia}
\descript{| \small Expected May 2022}
\begin{adjustwidth}{0.55cm}{1.25cm}
  \custombold{B.ASc Engineering Physics} \\
  \customit{Coursework includes} Lagrangian Mechanics, Computational Modelling, Digital Systems and Microcomputers, Signals and Systems, Applied Quantum Mechanics, Linear Algebra, Honours Multivariable and Vector Calculus, Complex Analysis, Optics, Statistical Mechanics
\end{adjustwidth}

\vspace{8pt}

\runsubsection{Denmark Technical University}
\descript{| \small Winter 2019 }
\begin{adjustwidth}{0.55cm}{1.25cm}
  \custombold{Exchange Semester} \\
  \customit{Coursework includes} Operating Systems, Deep Learning, Robotics, Computationally Hard Problems. \\
  Won DTU OS Course Competition for writing the fastest reverse hash server in C
\end{adjustwidth}

\vspace{10pt}

\end{document}
