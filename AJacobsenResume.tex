% Original template author:
% Debarghya Das (http://debarghyadas.com)
%
% Original repository:
% https://github.com/deedydas/Deedy-Resume
%
% Modifications and my personal history:
% Me

\documentclass[]{deedy-resume-openfont}
\setlength{\parskip}{.4em}
\thispagestyle{empty}

\begin{document}

\namesection{}{Axel Jacobsen}{\href{https://axel-jacobsen.github.io}{axel-jacobsen.github.io} | \href{https://linkedin.com/in/Axel-Jacobsen}{linkedin.com/in/Axel-Jacobsen} \\ \href{mailto:axelnjacobsen@gmail.com}{axelnjacobsen@gmail.com}}

\begin{minipage}[t]{\textwidth}
	\runsubsection{Programming} Python 5 years, C/C++ 2 years, Java/MATLAB 1 year\\
	\runsubsection{Electrical / Mechanical} Experience with PCB design, mechanical prototyping and CAD software, fluid system design \\
	\runsubsection{Other} Have EU passport (Citizen of Denmark, Canada)
\end{minipage}

\vspace{12pt}

\section{Experience}

\runsubsection{R\&D Engineering Intern}
\descript{| Chan-Zuckerberg Biohub | \small{Jun 2020 - Dec 2020}}
\customit{Reprogrammed the \href{https://opentrons.com/ot-2/}{Opentrons OT-2}, an open-source pipetting robot for lab automation}
\begin{tightemize}
    \item Rewrote OT-2 code base with a focus on speeding up epithelial cell growth protocols - doubled speed of certain frequent operations, reduced code base size by \char`\~ 75\% while expanding the capabilities of the robot
    \item Created computing architecture so arbitrary instruments (e.g. cell counters) can be used during protocols instead of just the Opentrons instruments
\end{tightemize}

\vspace{8pt}

\runsubsection{Engineering Intern}
\descript{| Wildlife Computers | \small{May 2019 – Aug 2019} }
\customit{Wildlife Computers is the leading provider of advanced wildlife telemetry solutions}
\begin{tightemize}
    \item Wrote C++ to test PCBs that arrive from fabrication - autonomously verifies PCB component placmeent to increase production throughput
    \item Designed an isolator PCB to isolate digital lines from sensitive measurement devices, allowing low-noise and accurate voltage measurements
\end{tightemize}

\vspace{8pt}

\runsubsection{Data Science Co-op}
\descript{| Control Mobile | \small{Jan 2018 – Apr 2018} }
\customit{Control Mobile aggregated and displayed transaction data for over 100 companies that used Stripe/Square/Paypal}
\begin{tightemize}
    \item Wrote Python scripts to analyze and rank order over 300 individual SQL queries by their runtime to optimize the SQL database; reduced the runtime to fetch and display customer data by 65\%
    \item Worked with the backend team to fix existing bugs, write new code, and refactor current code
\end{tightemize}

\vspace{12pt}


\section{Projects}

\runsubsection{Deep Learning}
\begin{tightemize}
    \item Asynchronous Advantage Actor-Critic Model written in Pytorch, optimized for multicore CPUs via multiprocessing
    \item LSTM-based Deep Q-Network, trained on Denmark Technical University's High-Performance Computing Cluster
    \item Feed-forward neural net written with Numpy solves MNIST with 97.2\% accuracy, vectorized for fast training
\end{tightemize}

\vspace{8pt}

\runsubsection{16-Week Autonomous Robot Competition}
\descript{| Engineering Physics\hfill}
\begin{tightemize}
    \item Deployed a real-time object detection algorithm on Raspberry Pi
    \item Created signal processing software to detect IR signals with sub-millisecond detection time
    \item Wrote C controls software and created circuits to control the mechanical subsystems
\end{tightemize}

\vspace{12pt}


\section{Education}

\runsubsection{University of British Columbia}
\descript{| \small Expected May 2022}
\begin{adjustwidth}{0.55cm}{1.25cm}
	\custombold{B.ASc Engineering Physics, GPA 3.70} \\
	\customit{Coursework includes} Lagrangian Mechanics, Computational Modelling (currently with fluids), Digital Systems and Microcomputers, Signals and Systems, Applied Quantum Mechanics, Linear Algebra, Honours Multivariable and Vector Calculus, Complex Analysis, Optics, Statistical Mechanics
\end{adjustwidth}

\vspace{8pt}

\runsubsection{Denmark Technical University}
\descript{| \small Winter 2019 }
\begin{adjustwidth}{0.55cm}{1.25cm}
	\custombold{Exchange Semester} \\
	\customit{Coursework includes} Operating Systems, Deep Learning, Robotics, Computationally Hard Problems. \\
	Won the DTU OS Challenge for writing the fastest reverse hash server in C, using both multiprocessing and multithreading 
\end{adjustwidth}

\vspace{10pt}

\end{document}
