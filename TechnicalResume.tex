
%%%%%%%%%%%%%%%%%%%%%%%%%%%%%%%%%%%%%%%
% This is a modified ONE COLUMN version of
% the following template:
% 
% Deedy - One Page Two Column Resume
% LaTeX Template
% Version 1.1 (30/4/2014)
%
% Original author:
% Debarghya Das (http://debarghyadas.com)
%
% Original repository:
% https://github.com/deedydas/Deedy-Resume
%
% IMPORTANT: THIS TEMPLATE NEEDS TO BE COMPILED WITH XeLaTeX
%
% This template uses several fonts not included with Windows/Linux by
% default. If you get compilation errors saying a font is missing, find the line
% on which the font is used and either change it to a font included with your
% operating system or comment the line out to use the default font.
% 
%%%%%%%%%%%%%%%%%%%%%%%%%%%%%%%%%%%%%%

% !TEX program = xelatex

\documentclass[]{deedy-resume-openfont}
    \setlength{\parskip}{.4em}

\begin{document}

\namesection{}{Axel Jacobsen}{\urlstyle{same}\url{axel-jacobsen.github.io} | \urlstyle{same}\url{linkedin.com/in/Axel-Jacobsen} \\
	\href{mailto:axelnjacobsen@gmail.com}{axelnjacobsen@gmail.com} | 778.789.4810}

	\section{Projects}
	\runsubsection{Sensor Team Lead}
	\descript{| UBC Subbots \hfill \small{Sept 2018 – Present}}
	\customit{Constructing an autonomous robotic submarine to compete in the \href{https://www.robonation.org/competition/robosub}{22nd International RoboSub Competition} }
	\begin{tightemize}
		\item Used Matlab to calibrate cameras for a computer vision algorithm to navigate the course
		\item Researched, tested, and documented the viability of pressure sensors and motor controllers for the robot
		\item Designed and created signal filtering and amplification circuits to prepare sensor data for further analysis
	\end{tightemize}
	\vspace{8pt}
	
	\runsubsection{Robot Competition}
	\descript{| Engineering Physics \hfill \small{May 2018 – Aug 2018}}
	\customit{Designed and built a robot to autonomously retrieve objects on a pre-built course}
	\begin{tightemize}
		\item First team in the history of the competition to use a neural network; used for locating objects on the course
		\item Designed and tested a custom IR distance sensor, servo control circuits, and an IR signal circuit
		\item Wrote main control software in C on an ARM STM32 to control the robot
		\item Wrote C software to digitally filter IR sine waves for 10 kHz frequency \\
		See \href{https://axel-jacobsen.github.io/ENPHRobot/}{axel-jacobsen.github.io/ENPHRobot/}
	\end{tightemize}
	\vspace{10pt}
	
	\section{Experience}
	
	\runsubsection{Data Science Co-op}
	\descript{| Control Mobile \hfill \small{Jan 2018 – Apr 2018} }
	\customit{Control Mobile aggregated and displayed transaction data for over 100 companies that used Stripe/Square/Paypal}
	\begin{tightemize}
		\item Wrote Python scripts to analyze and rank order over 300 individual SQL queries by their runtime in order \\ to systematically optimize the SQL database; reduced the runtime to fetch and display customer data by 65\%
		\item Worked with the agile backend team to fix existing bugs, write new code, and to refactor current code
		\item Fixed security issues that would leave the website vulnerable to SQL injection attacks
	\end{tightemize}
	\vspace{8pt}
	
	\runsubsection{Junior Software Developer}
	\descript{| ubyssey.ca \hfill \small{May 2017 – Aug 2017 }}
	\customit{The Ubyssey is the campus newspaper for the University of British Columbia}
	\begin{tightemize}
		\item Wrote Python/Javascript code for Dispatch, the publishing platform for The Ubyssey
		\item Created Django and React UI for Dispatch that allows content to be written and uploaded to \\ the website by non-technical users such as editors, writers
		\item Refactored old Django code to current best practices for security, readability and reliability
	\end{tightemize}
	\vspace{10pt}
	
	\section{Volunteer Experience}
	\runsubsection{Engineering Physics Mentor}
	\descript{| UBC \hfill \small{Sept 2018 – Present}}
	\begin{tightemize}
		\item Mentor of five 2nd year Engineering Physics Students
	\end{tightemize}
	\vspace{8pt}
	
	\runsubsection{Squadron Commander}
	\descript{| 103 Thunderbird Squadron, Royal Canadian Air Cadets \hfill \small{Apr 2015 – July 2016}}
	\begin{tightemize}
		\item Leader of a squadron of 80 cadets
		\item In charge of weekly squadron meetings, mentoring senior cadets and enforcing standards of leadership and citizenship of the squadron
	\end{tightemize}
	\vspace{10pt}
	
	\section{Education}
	\runsubsection{University of British Columbia}
	\descript{ \hfill \small Expected May 2021}
	\begin{tightemize}
		\item Bachelor of Applied Science, Engineering Physics \\
	\end{tightemize}
	\vspace{8pt}
	
	\section{About Me}
	\begin{minipage}[t]{.35\textwidth}
		\runsubsection{Programming Languages} \\
		Python \textbullet{} C \textbullet{} Java \\
		JavaScript \textbullet{} MATLAB \\
		\LaTeX
		\vspace{8pt}
	\end{minipage}
	\hfill
	\begin{minipage}[t]{.55\textwidth}
		\runsubsection{Summary} \\
		I am an enthusiastic Engineering Physics Student at the University of British Columbia, with a passion for mathematics, physics, and robotics. I spend my free time working on my personal projects, climbing, biking, or skiing.
	\end{minipage}
	

\end{document}  \documentclass[]{article}