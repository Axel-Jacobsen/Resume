%%%%%%%%%%%%%%%%%%%%%%%%%%%%%%%%%%%%%%%
% This is a modified ONE COLUMN version of
% the following template:
% 
% Deedy - One Page Two Column Resume
% LaTeX Template
% Version 1.1 (30/4/2014)
%
% Original author:
% Debarghya Das (http://debarghyadas.com)
%
% Original repository:
% https://github.com/deedydas/Deedy-Resume
%
% This template uses several fonts not included with Windows/Linux by
% default. If you get compilation errors saying a font is missing, find the line
% on which the font is used and either change it to a font included with your
% operating system or comment the line out to use the default font.
% 
%%%%%%%%%%%%%%%%%%%%%%%%%%%%%%%%%%%%%%

% !TEX program = xelatex

\documentclass[]{deedy-resume-openfont}
    \setlength{\parskip}{.4em}

\begin{document}

\headersection{Axel Jacobsen}{header.png}

\section{Experience}

\runsubsection{Data Science Co-op}
\descript{| Control \hfill \small{Jan 2018 – Apr 2018} }
\begin{tightemize}
	\item Optimized SQL queries to increase execution speed by 283\%
	\item Wrote and Re-factored Javascript for the website’s API for concice and correct code
\end{tightemize}
\sectionsep

\runsubsection{Junior Software Developer}
\descript{| ubyssey.ca \hfill \small{May 2017 – Aug 2017 }}
\begin{tightemize}
	\item Worked on Dispatch (Django / React backend platform of \textit{ubyssey.ca})
    \item Created components that allows content to be modified by non-technical employees
    \vspace{10pt}
\end{tightemize}
\sectionsep

\section{Projects}
\runsubsection{Sensor Team Lead}
\descript{| UBC Subbots \hfill \small{Sept 2018 – Present}}
\begin{tightemize}
	\item Team is constructing an autonomous robotic submarine to compete in \\ the \href{https://www.robonation.org/competition/robosub}{22nd International RoboSub Competition}
	\item Responsible for  integrating all sensors on the robot into ROS framework including cameras, \\ depth sensor, IMU, and hydrophones
	\item Designing signal amplification, filtering circuits, and processing signal for relevant information
\end{tightemize}
\sectionsep

\runsubsection{Robot Competition}
\descript{| Engineering Physics \hfill \small{May 2018 – Aug 2018}}
\begin{tightemize}
	\item Designed and built a robot to autonomously navigate and retrieve yarn Ewoks
	\item We located the Ewoks by running the \href{https://pjreddie.com/darknet/yolov2/}{YoloV2} neural network on a Raspberry Pi
	\item Developed custom IR distance sensor, servo control circuits, and IR signal circuit
	\item Wrote software to quickly and accurately detect 1 kHz and 10 kHz signals \\
	See \href{https://axel-jacobsen.github.io/ENPHRobot/}{axel-jacobsen.github.io/ENPHRobot/}
	\vspace{10pt}
\end{tightemize}
\sectionsep

\section{Volunteer Experience}
\runsubsection{Engineering Physics Mentor}
\descript{| UBC \hfill \small{Sept 2018 – Present}}
\begin{tightemize}
	\item Mentor of five 2nd year Engineering Physics Students
\end{tightemize}
\sectionsep

\runsubsection{Squadron Commander}
\descript{| 103 Thunderbird Squadron \hfill \small{Apr 2015 – July 2016}}
\begin{tightemize}
	\item Led the Command Team, a group responsible for running the Squadron of  20 senior \\ and 60  junior cadets
    \vspace{10pt}
\end{tightemize}
\sectionsep


\section{Education}
\runsubsection{University of British Columbia}
\descript{ \hfill \small Expected May 2021}
\begin{tightemize}
	\item Bachelor of Applied Science, Engineering Physics | Minor in Mathematics \\
\end{tightemize}
\sectionsep

\section{About Me}
\begin{minipage}[t]{.35\textwidth}
	\subsection{Programming Languages}
	Python \textbullet{} C \textbullet{} Java \\
	JavaScript \textbullet{} MATLAB \\
	\LaTeX
	\sectionsep
\end{minipage}
\hfill
\begin{minipage}[t]{.55\textwidth}
	\subsection{Summary}
	I am an enthusiastic Engineering Physics Student at the University of British Columbia, with passion for mathematics, programming, and robotics. I spend my free time working on my personal projects, climbing, biking, or skiing.

	\vspace{10pt}

	% \namesectionnounderline{}{Further Information}{ \urlstyle{same}\url{github.com/Axel-Jacobsen} | \urlstyle{same}\url{linkedin/in/Axel-Jacobsen} \\
	% 	\href{mailto:axelnjacoben@gmail.com}{axelnjacoben@gmail.com} | 778.789.4810
	% }
\end{minipage}

\end{document}  \documentclass[]{article}